\section{Introducción}

El presente documento tiene como objetivo documentar el desarrollo del primer
proyecto programado del curso de Lenguajes de Programación, impartido por el
profesor Oscar Viquez Acuña, en el Instituto Tecnológico de Costa Rica.

El proyecto consiste en el desarrollo de un servidor de música, el cual debe
permitir a los clientes conectarse y reproducir música, así como también
administrar la lista de canciones disponibles para reproducir, todo esto 
de manera concurrente, utilizando el lenguaje de programación Go y las ventajas
del paradigma imperativo.

Además, se debe implementar un cliente que permita conectarse al servidor y
reproducir la música que este ofrece, teniendo la posibilidad de administrar
listas de reproducción de manera local, utilizando el lenguaje de programación
F\# y las ventajas del paradigma funcional.

A lo largo de este documento se detallarán los aspectos más importantes del
desarrollo de este proyecto, así como también se discutirán los hallazgos
pertinentes a la implementación de diversas tecnologías y funcionalidades
requeridas para un correcto y eficiente funcionamiento del servidor y el
cliente.