\section{Conclusiones y Recomendaciones}

Durante el desarrollo del proyecto se implementaron dos programas, un servidor
y un cliente. Cada uno cuenta con características y funcionalidades distintas,
con el fin de cumplir con los requerimientos establecidos para cada uno, en varias 
ocasiones se decidió optar por alternativas que no eran las indicadas, pero que sin 
embargo según el criterio del equipo de desarrollo, eran las más adecuadas para
lograr obtener un producto final que cumpliera con los requerimientos establecidos.

Respondiendo a la pregunta de qué elementos consideran, deben tomarse en cuenta
mejorar/cambiar, con miras en una implementación robusta de un sistema similar
en producción a gran escala, se puede mencionar que el hecho de utilizar 
protocolos más adecuados para cada caso de uso permitiría mejorar el rendimiento
del sistema. Tal como se hizo al implementar el protocolo HLS. Al mismo tiempo, 
también se considera oportuno el hecho de utilizar frameworks y herramientas 
que permitan la implementación de sistemas de manera más sencilla y eficiente,
como lo puede ser Go por su concurrencia y el framework de Avalonia por 
su buena documentación y multiplataforma.

Por último, se puede concluir que el desarrollo de este proyecto permitió
aplicar los conocimientos adquiridos durante el curso de Lenguajes de
Programación, manejando un lenguaje de tipo imperativo/concurrente y datos 
inmutables por medio de un lenguaje funcional, así como también permitió
aprender nuevas tecnologías y herramientas que no se habían utilizado
anteriormente. De esta manera, se logró cumplir con los objetivos planteados
para el proyecto de manera exitosa.